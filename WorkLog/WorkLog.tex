
\documentclass[10]{article}
\usepackage{graphicx}
\begin{document}
\title{Bare-Metal Development \\
M2M Lectures\\
Grenoble University}
\author{François-Xavier Gros, Jahna Neola}
\date{\today}
\maketitle

\section{Preface by Pr. Olivier Gruber}

This document is your work log for the first step in the 
M2M course, master-level, at the University of Grenoble, France.
You will have such a document for each step of our course
together.

This document has two parts. One part is about diverse
sections, each with a bunch of questions 
that you have to answers. The other part is really 
a laboratory log, keeping track of what you do, 
as you do it.

The questions provide a guideline for your learning. 
They are not about getting a good grade if you answer them
correctly, they are about giving your pointers on what to 
learn about.

The goal of the questions is therefore not to be answered 
in three lines of text and be forgotten about. The questions
must be researched and thoroughly understood. Ask questions
around you if things are unclear, to your fellow students
and to me, your professor. 

Writing down the answers to the questions is a tool for helping
your learn and remember. Also, it keeps track of what you know,
the URLs you visited, the open questions that you are trouble with,
etc. The tools you used. It is intended to be a living document,
written as you go.

Ultimately, the goal of the document is to be kept for 
your personal records. If ever you will work on embedded
systems, trust me, you will be glad to have a written 
trace about all this.

{\bf REMEMBER:} plaggia is a crime that can get you evicted
forever from french universities... The solution is simple,
write using your own words or quote, giving the source of
the quoted text. Also, remember that you do not learn through
cut\&paste. You also do not learn much by watching somebody else
doing.


\section{QEMU}

QEMU est un émulateur, il représente la puce/carte + le cable + le terminal dédié (configuré pour l'interaction avec le hardware spécifique).

Read the README-QEMU-ARM in the project ``arm.boot'' (workspace/arm.boot)

Avec `make run`, QEMU est appelé avec les options suivantes

\begin{itemize}
  \item -M VersatileAB : type de machine à émuler
  \item -m 1M : la quantité de mémoire du matériel
  \item -nographic : sans interface graphique
  \item -kernel kernel.bin : le binaire qu'execute 
  \item -serial mon:stdio : sortie standard utilisée
  \item -gdb tcp::1234 : configuration pour utiliser gdb
  \item -S : "freeze CPU at startup" : exige de lancer explicitement l'execution avec s
\end{itemize}

Les deux derniers options sont spécifiques au mode debug car il s'agit de la configuration gdb et de permettre à l'utilisateur de lancer l'exécution.

Différence entre fichier ELF et fichier BIN :

Fichier ELF comprend le fichier BIN plus des métadonnées permettant de débugger le programme (variables de debuggage, des informations sur la mémoire, table des symboles...).

\section{GNU Debugger}

GNU s'exécute avec le .elf. A besoin d'être configuré pour se lancer sur le hardward, ici l'emulateur QEMU.

How early are you in the boot sequence of the hardware when
executing the first instruction? What software runs before
the first instruction of our code (kernel.bin)?

hardware se lance avec startup.s avec PC placé au stack_top (rien n'indique des commandes soient déjà écrites à cet instant ) puis le point d'entrée dans le programme C est entré (`c_entry' est alors considéré comme un main). Le fichier kernel n'est exécuté que lorsque la machine est completement démarée et bien configurée.

\section{Makefile}

You need to read and fully understand the provided makefile.
Please find a few questions below highlighting important points
of that makefile. These questions are there only to guide your
reading of the makefile. Make sure they are addressed in your
overall writing about the makefile and the corresponding challenge
of building bare-metal software.

\begin{enumerate}
\item 
  What is the TOOLCHAIN?
  Ensemble compatible de programmes selon leurs interfaces et leurs specifications, ensemble ils peuvent compiler, deassembler, linker et convertir des programmes.
\item
  What is the meaning of "none" and "eabi" in the compiler
  name "arm-none-eabi-gcc"?
  none est que c'est qu'on vise une execution "bare-metal" sans passer par un OS
  eabi est la première couche entre le kernel Linux et la distribution 
\item
  What are VersatilePB and VersatileAB?
  Ce sont deux types de cartes polyvalentes. La premiere est PlatformBaseboard : bon pour developpement sur système embarqué avec PCI (slot pour carte graphique par ex). La seconde est Application Baseboard : bon pour les applications mobiles.
\item
  What is a linker script? Look at the linker option "-T"
  Notre linker : arm-none-eabi-ld, à partir de fichier ELF (binaire) créé un nouveau fichier ELF regroupant l'execution totale du programme
  - T FILE : spécifie un fichier qui sera le script pour linker 
  ici on utilise kernel.ld qui gère la configuration de base de kernel.elf (le binaire de retour) dont la configuration nécessaire à startup, et les definitions des zones memoire
\item
  Why do we translate the "kernel.elf" into a "kernel.bin" via "objcopy"?
  objcopy autorise la modification du binaire durant sa copie, le flag -O binary est une méthode pour specifier le format de retour, vers un binaire brute tous les symboles et informations de reallocation sont enlevés.
\item
  What compiler/linker options ensure that we can debug?
  2 : -gdb tcp::1234 -S : le premier pour lier gdb au programme de la carte emulée, le second pour stoper l'execution initiale et donner la main à l'utilisateur
\item
  What is the meaning of the "-nostdlib" option? Why is it necessary?
  Indique que rien ne peut être recuperer dynamiquement, tout doit être dans le binaire.
\item
  Try MEMORY=32K, it fails, why? Look at the linker script.
  Erreur : pas de fails au build mais les uart_send_string et uart_receive ne fonctionnent pas. 0x10000 superieur a 32K donc methode d'entree non definie
\item
  Could you use printf in the code? Why?
  No, it is a part of libc that is not included
\end{enumerate}

\subsection{Linker Script}

Detail here your understanding of the linker script that we use.

Linker script :

\begin{itemize}
  \item definit le point d'entrée (ENTRY) ici startup.s
  \item definit les sections de la memoire
  \item comprend la stack, le bss (variables globales, mémoire mise à 0 par sureté), data
  \item addresse 0x10000 car le kernel Qemu s'y enregistre par défaut (plus efficace qu'une reallocation dans notre cas)
\end{itemize}

\subsection{ELF Format}

How does the ELF executable contain debug information? Which option must
be given to the compiler and linker? Why to both?

Cette fonctionne donne des informations sur un ou plusieurs fichiers, ELF est un executable qui comprend ce que comprendrais un bin + des informations dans une table qui permettent notamment le debeuggage. 

kernel.elf comprend des section d'assembleur par fonction + des variables de debug (etat courant notament) en table de symboles 

file renvoit
ELF 32-bit LSB executable, ARM, EABIS v 1 (SYSV) statically linked, with debug_info, not stripped

\section{Startup Code}

Point d'entrée du programme, lien avec le code C, initialise les registres.

\section{Main Code}

3 méthodes : send/receive a char et send_string.

c_entry : la méthode main qui est executé à l'appel du fichier. 

Cette méthode

\begin{itemize}
  \item affiche les 2 premieres string
  \item fait une boucle infinie
  \item tant qu'on a pas reçu de caractere pendant un laps de temps, on envoie "zzz" si ECHO_ZZZ sinon rien
  \item si le caratere existe on l'affiche dans la console (comportement particulier pour le retour chariot)
\end{itemize}

Chaque caractere apparait dès qu'il est tapé


\begin{enumerate}
\item
  What is an UART and a serial line? 
  UART pin d'entrée sur la board, serial line = line serie connecteur fillaire entre la board et les peripheriques
\item
  What is the purpose of a serial line here?
  emulation de la ligne serie associée à la board, gère les interactions utilisateur (entrée - sortie), ici on l'a connectée a stdio (les standard) ce qui permet d'interagir avec le clavier et l'ecran
\item
  What is the relationship between this serial line
  and the Terminal window running a shell on your laptop?
\item
  What is the special testing of the value 13 as a special
  character and why do we send back '\textbackslash r' and '\textbackslash n'?
  13 est le code ASCII du retour chariot (uniforme à tous les OS) c'est donc ce code qu'on teste plutôt qu'une valeur/char qui dépend de l'OS
\item
  Why can we say this program polls the serial line?
  Although it works, why is it not a good idea?
  Boucle avec le while sur le caractere d'entrée : sur-utilisation des ressources 
\item
  How could using hardware interrupts be a better solution?
  libère le processeur pour de vraies activités (pas attendre)
\item
  Could we say that the function uart\_send may block? why?
  Oui, si la queue UART TX est pleine.
\item
  Could we say that the function uart\_receive is non-blocking? why?
  Oui, la queue est en entrée, elle peut être vide ou pleine.
\item
  Explain why uart\_send is blocking and uart\_receive is non-blocking.  
\end{enumerate}

\section{Test Code -- TODO}

\subsection{Blocking Uart-Receive}

Change the code so that the function uart\_receive is blocking.

Why does it work in this particular test code?

Why would it be an interesting change in this particular setting?

\subsection{Adding Printing}

We provided you with the code of a kernel-version of printf,
the function called "kprintf" in the file "kprintf.c".

Add it to the makefile so that it is compiled and linked in.

Look at the function "kprintf" and "putchar" in the file
"kprintf.c". Why is the function "putchar" calling the
function "uart\_send"?

Use the function kprintf to actually print the code of the
characters you type and not the characters themselves.

Hit the following special keys:

\begin{itemize}
\item left and right arrow.
\item backspace and delete key.
\end{itemize}

Explain what you see.
      
\subsection{Line editing}

The idea is now to allow the editing of the current line:

\begin{itemize}
\item Using the left and right arrows
\item Using the "backspace" and "delete" keys 
\end{itemize}

First, experiment using the left/right arrows...
and the backspace/delete keys... 

\begin{itemize}
\item Explain what you see
\item Explain what is happening?
\end{itemize}

Now that you understand, write the code to allow for line editing.

\section{Laboratory Log}

\end{document}

