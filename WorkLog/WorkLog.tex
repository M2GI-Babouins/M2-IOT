
\documentclass[10]{article}
\usepackage{graphicx}
\begin{document}
\title{Bare-Metal Development \\
M2M Lectures\\
Grenoble University}
\author{Your Names Here}
\date{\today}
\maketitle

\section{Preface by Pr. Olivier Gruber}

This document is your work log for the first step in the 
M2M course, master-level, at the University of Grenoble, France.
You will have such a document for each step of our course
together.

This document has two parts. One part is about diverse
sections, each with a bunch of questions 
that you have to answers. The other part is really 
a laboratory log, keeping track of what you do, 
as you do it.

The questions provide a guideline for your learning. 
They are not about getting a good grade if you answer them
correctly, they are about giving your pointers on what to 
learn about.

The goal of the questions is therefore not to be answered 
in three lines of text and be forgotten about. The questions
must be researched and thoroughly understood. Ask questions
around you if things are unclear, to your fellow students
and to me, your professor. 

Writing down the answers to the questions is a tool for helping
your learn and remember. Also, it keeps track of what you know,
the URLs you visited, the open questions that you are trouble with,
etc. The tools you used. It is intended to be a living document,
written as you go.

Ultimately, the goal of the document is to be kept for 
your personal records. If ever you will work on embedded
systems, trust me, you will be glad to have a written 
trace about all this.

{\bf REMEMBER:} plaggia is a crime that can get you evicted
forever from french universities... The solution is simple,
write using your own words or quote, giving the source of
the quoted text. Also, remember that you do not learn through
cut\&paste. You also do not learn much by watching somebody else
doing.

\section{QEMU}

What is QEMU?
Why is it necessary here?

Read the README-QEMU-ARM in the project ``arm.boot'' (workspace/arm.boot)

Try it... with "make run". Understand the options given to QEMU.

Relate the kernel option and the makefile and the difference between the ELF file (kernel.elf) and the binary file (kernel.bin) produces by the makefile.

\section{GNU Debugger}

Here, try single-stepping the code via the GNU debugger (GDB).

Read the README-GDB (workspace/arm.boot) and the GDB cheat sheet.

Try it, in one shell, launch QEMU in debug mode (make debug)
and in another shell launch gdb. Relate the options given to QEMU
and the way you launch gdb and connect to QEMU.

How early are you in the boot sequence of the hardware when
executing the first instruction? What software runs before
the first instruction of our code (kernel.bin)?

\section{Makefile}

You need to read and fully understand the provided makefile.
Please find a few questions below highlighting important points
of that makefile. These questions are there only to guide your
reading of the makefile. Make sure they are addressed in your
overall writing about the makefile and the corresponding challenge
of building bare-metal software.

\begin{enumerate}
\item 
  What is the TOOLCHAIN?
\item
  What is the meaning of "none" and "eabi" in the compiler
  name "arm-none-eabi-gcc"?
\item
  What are VersatilePB and VersatileAB?
\item
  What is a linker script? Look at the linker option "-T"
\item
  Read and understand the linker script that we use
\item
  Why do we translate the "kernel.elf" into a "kernel.bin" via "objcopy"  
\item
  What compiler/linker options ensure that we can debug?
\item
  What is the meaning of the "-nostdlib" option? Why is it necessary?
\item
  Try MEMORY=32K, it fails, why? Look at the linker script.
\item
  Could you use printf in the code? Why?
\end{enumerate}

\subsection{Linker Script}

Detail here your understanding of the linker script that we use.

Why do we translate the "kernel.elf" into a "kernel.bin" via "objcopy"

Why do we link our code to run at the 0x10000? 

Why do we make sure the code for the object file "startup.o" is first?

\subsection{ELF Format}

What is the ELF format?

Look at the command "arm-none-eabi-objdump"

Why is it used as an object file format and an executable file format.

How does the ELF executable contain debug information? Which option must
be given to the compiler and linker? Why to both?

Confirm what ELF object files and the final ELF executable are with the shell command "file".

Look at the ELF object files and the final ELF executable with the tool: arm-none-eabi-objdump.

\section{Startup Code}

Read and understand the startup code in the file "startup.s".

Explain here what it does.

\section{Main Code}

Read and understand the main code in the file "main.c".

Explain here what it does. In particular, explains how the characters
you type in the terminal window actually appear on the terminal window.

With a regular shell, the shell echoes the character as you type them.
It only sends the characters once you hit the return key, as a complete line.

It is the behavior you notice here? 

\begin{enumerate}
\item
  What is an UART and a serial line? 
\item
  What is the purpose of a serial line here?
\item
  What is the relationship between this serial line
  and the Terminal window running a shell on your laptop?
\item
  What is the special testing of the value 13 as a special
  character and why do we send back '\textbackslash r' and '\textbackslash n'?
\item
  Why can we say this program polls the serial line?
  Although it works, why is it not a good idea?
\item
  How could using hardware interrupts be a better solution?
\item
  Could we say that the function uart\_send may block? why?
\item
  Could we say that the function uart\_receive is non-blocking? why?
\item
  Explain why uart\_send is blocking and uart\_receive is non-blocking.  
\end{enumerate}

\section{Test Code -- TODO}

\subsection{Blocking Uart-Receive}

Change the code so that the function uart\_receive is blocking.

Why does it work in this particular test code?

Why would it be an interesting change in this particular setting?

\subsection{Adding Printing}

We provided you with the code of a kernel-version of printf,
the function called "kprintf" in the file "kprintf.c".

Add it to the makefile so that it is compiled and linked in.

Look at the function "kprintf" and "putchar" in the file
"kprintf.c". Why is the function "putchar" calling the
function "uart\_send"?

Use the function kprintf to actually print the code of the
characters you type and not the characters themselves.

Hit the following special keys:

\begin{itemize}
\item left and right arrow.
\item backspace and delete key.
\end{itemize}

Explain what you see.
      
\subsection{Line editing}

The idea is now to allow the editing of the current line:

\begin{itemize}
\item Using the left and right arrows
\item Using the "backspace" and "delete" keys 
\end{itemize}

First, experiment using the left/right arrows...
and the backspace/delete keys... 

\begin{itemize}
\item Explain what you see
\item Explain what is happening?
\end{itemize}

Now that you understand, write the code to allow for line editing.

\section{Laboratory Log}

\end{document}

